\input{preamble.tex}
\begin{document}
\large
\maketitle
\newpage
\begin{mainpage}{0}{大綱}
\vfill
\begin{multicols}{2}
\large
\begin{itemize}
\item 在踏入開源之前
\item 在接觸程式語言之後
\item 初踏開源
\end{itemize}
\begin{itemize}
\item 在程式之外
\item 寫書
\end{itemize}
\end{multicols}
\mbox{ }
\end{mainpage}
%
%%%%%%%% 在踏入開源之前 %%%%%%%%
%
\newpage
\mytitle{在踏入開源之前}
\newpage
\begin{tikzpage}{1}{自我介紹}
\begin{tikzpicture}[x= 1mm, y = 1mm]
\draw (-60, -35) rectangle (60, 35);
\draw (-50, -25) rectangle (-10, 25);
\node[right] at (-5, 0) {
\begin{minipage}{60mm}
關於我:
\begin{itemize}
\normalsize
\item 姓名:Si manglam/周造麟
\item 蘭嶼人
\item 剛剛學會寫程式的準大一
\end{itemize}
\end{minipage}
};
\end{tikzpicture}
\end{tikzpage}
\newpage
\begin{tikzpage}{1}{自我介紹}
\begin{tikzpicture}[x= 1mm, y = 1mm]
\draw (-60, -35) rectangle (60, 35);
\draw (-50, -25) rectangle (-10, 25);
\node[right] at (-5, 0) {
\begin{minipage}{60mm}
關於我:
\begin{itemize}
\normalsize
\item 喜歡研究的事情
\item 想辦法改善遇到的問題
\item 喜歡奇怪的事情
\end{itemize}
\end{minipage}
};
\end{tikzpicture}
\end{tikzpage}
\newpage
\begin{tikzpage}{1}{自我介紹}
\begin{tikzpicture}[x= 1mm, y = 1mm, grow cyclic,
	level 1/.style={level distance=2.5cm,sibling angle=120, fill = mygreen},
	level 2/.style={level distance=3 cm,sibling angle=45}]
\draw[draw = none, fill = none] (-60, -35) rectangle (60, 35);
\node[fill = mygreen, circle] {高一生活}
	child{node[fill = mygreen, circle] {化學}}
	child{node[fill = mygreen, circle] {化學}}
	child{node[fill = mygreen, circle] {化學}}
	;
\end{tikzpicture}
\end{tikzpage}
\newpage
\mysubtitle{要怎麼打出化學方程式?}
\newpage
\mysubtitle{\scalebox{3}{\LaTeX}}
%
%%%%%%%% 接觸程式語言之後 %%%%%%%%
%
\newpage
\mytitle{程式語言}
\newpage
\begin{mainpage}{2}{LaTeX}
\mbox{ }
\mysubtitle{\scalebox{3}{\LaTeX ?}}
\end{mainpage}
\newpage
\begin{mainpage}{2}{LaTeX}
{\LARGE \LaTeX:}\\
\begin{itemize}
\setlength{\itemindent}{3em}
\item 標記語言
\item 經過 \TeX 的編譯排版成為精緻的文件
\end{itemize}
\end{mainpage}
\newpage
\begin{mainpage}{2}{la}
LuaLaTeX 介紹
\end{mainpage}
\newpage
\begin{mainpage}{2}{LuaLaTeX}
\mysubtitle{
\begin{tikzpicture}[scale = 2]
\node (latex) {\LaTeX};
\node[node distance = 4cm, right of = latex] (lua) {Lua};
\draw[-{Stealth[length=5mm]}] (latex)--(lua);
\end{tikzpicture}
}
\end{mainpage}
\newpage
\begin{mainpage}{2}{LuaLaTeX}
\mysubtitle{
\begin{tikzpicture}
\node (latex) {\LaTeX};
\node[node distance = 4cm, right of = latex] (lua) {Lua};
\node[scale = 2] at($(latex)!0.5!(lua)$) {X};
\draw[-{Stealth[length=5mm]}] (latex)--(lua);
\end{tikzpicture}
}
\end{mainpage}
\newpage
\begin{mainpage}{2}{動畫}
\setlength{\leftskip}{0mm}
\setlength{\rightskip}{0mm}
\begin{center}
\vskip-12pt
\vfill
\animategraphics[draft]{72}{test-figure}{0}{720}
\vfill
\vskip-12pt
\end{center}
\end{mainpage}
\newpage
%% 放其他動畫圖
\begin{mainpage}{2}{資料}
放其他動畫圖
\end{mainpage}
\newpage
\begin{tikzpage}{2}{資料}
\begin{tikzpicture}[x = 1 mm, y = 1 mm, node distance = 15mm and 15mm]
\draw[draw = none, fill = none] (-60, -35) rectangle (60, 35);
\coordinate (a) at (0, 35);
\node[text width=14pt, below, execute at begin node=\setlength{\baselineskip}{1pt}] (b) at (-20,20) {老舊繁體中文資料};
\node[text width=14pt, below, execute at begin node=\setlength{\baselineskip}{1pt}] (c) at (0,20) {中等簡體中文資料};
\node[text width=14pt, below, execute at begin node=\setlength{\baselineskip}{1pt}] (d) at (20,20) {最新英文資料};
\draw (a)--(b.north);
\draw (a)--(c.north);
\draw (a)--(d.north);
\end{tikzpicture}
\end{tikzpage}
\newpage
\mysubtitle{自主學習主題!!}
\newpage
\mytitle{初踏開源}
\newpage
%
%%%%%%%% 自主學習 %%%%%%%%
%
\begin{tikzpage}{3}{自主學習}
\begin{tikzpicture}[x = 1 mm, y = 1 mm]
\draw (-60, -35) rectangle (60, 35);
\draw (-45, 15) pic{mynode ={提案}};
\draw (0, -15) pic{upnode ={檢核}};
\draw (45, 15) pic{mynode ={檢核}};
\draw[->, line width = 1mm] (-45, 0)--(45, 0);
\end{tikzpicture}
\end{tikzpage}
\newpage
\begin{mainpage}{3}{TeXmaker}
\begin{center}
\includegraphics[width = 60mm]{動畫測試/texmaker_icon.jpg}
\end{center}
\end{mainpage}
\newpage
\begin{mainpage}{3}{Translation}
\mysubtitle{英文 > 中文}
\end{mainpage}
\newpage
\begin{mainpage}{3}{Translation}
\mysubtitle{英文 < 中文}
\end{mainpage}
\newpage
\begin{mainpage}{3}{TeXmaker}
TeXmaker 的翻譯過程動畫 Part 1
\end{mainpage}
\newpage
\begin{mainpage}{3}{Qt}
\begin{center}
\includegraphics[width = 90 mm]{動畫測試/qt.jpeg}
\end{center}
\end{mainpage}
\newpage
\begin{mainpage}{3}{TeXmaker}
\mysubtitle{TeXmaker is open source}
\end{mainpage}
\newpage
\begin{mainpage}{3}{TeXmaker}
TeXmaker 的翻譯過程動畫 Part 2
\end{mainpage}
\newpage
\mytitle{在程式之外}
\newpage
\mysubtitle{開源還可以是什麼}
\newpage
%
%%%%%%%% 程式之外 %%%%%%%%
%
\begin{mainpage}{4}{開源還能是什麼}
\mysubtitle{程式?原始碼?}
\end{mainpage}
\newpage
\begin{mainpage}{4}{開源還能是什麼}
\mysubtitle{程式\\\rotatebox{90}{=}\\一組給電腦的、用來解決某種問題的指令。}
\end{mainpage}
\newpage
\begin{mainpage}{4}{開源還能是什麼}
關鍵字顯示動畫
\mysubtitle{程式\\\rotatebox{90}{=}\\一組給電腦的、用來解決某種問題的指令。}
\end{mainpage}
\newpage
\begin{mainpage}{4}{開源還能是什麼}
\mysubtitle{原始碼\\\rotatebox{90}{=}\\???????????}
\end{mainpage}
\newpage
\begin{mainpage}{4}{開源還能是什麼}
\mysubtitle{原始碼\\\rotatebox{90}{=}\\???????????}
\end{mainpage}
\newpage
\begin{mainpage}{4}{開源還能是什麼}
\mysubtitle{廣義的開源}
\end{mainpage}
\newpage
%
%%%%%%%% 寫書 %%%%%%%%
%
\mytitle{寫書}
\newpage


\mysubtitle{結語}

\end{document}